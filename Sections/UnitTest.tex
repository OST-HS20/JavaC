\section{Testen}
Ofte werden für UnitTests Äuivalenzklassen gebildet welche Parameter in Bereich zerlegen, die von der Funktion wahrscheinlich gleich behandelt werden. Für jeden Bereich eine Eingangsvariable wählen und Testfall schreiben.

\subsection{JUnit Framework}
Die Testmethoden in JUnit haben keine Parameter und immer den Rückgabetype void. Sie sind mit der Annotation \textit{org.junit.jupiter.api.\textbf{@Test}} deklariert.

\begin{lstlisting}
@Test
void testMethod() {
	int negativeValue = -1;
	assertEquals(1, abs(negativeValue));	
}
\end{lstlisting}

Wichtige Asserts:
\begin{itemize}[nosep]
	\item assertEquals(expected, actual)
	\item assertEquals(expected, actual, delta)
	\item assertArrayEquals(expected, actual)
	\item assertTrue(actual)
\end{itemize}