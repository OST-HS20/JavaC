\section{IO}
Grundsätzlich werden IO-Streams mit Byte Stream \textit{InputStram, OutputStream} mit zB \textit{FileInputStream, FileOutputStream} erzeugt. Für UTF-16 können konkret die \textit{Reader, Writer} für Zeichen-/Zeilenweise Ein-\& Ausgabe verwerndet werden.

\noindent Java bietet auch direkten Daten-Zugriff durch die statische Klasse Files an. \textbf{Achtung:} Benötigen IOException!
\begin{itemize}[nosep]
	\item Files.readAllBytes(Path.of(str))
	\item Files.write(Path.of(str), data)
	\item Files.readAllLines(Path.of(str), StandardCharsets.UTF\_8) : List[String]
	\item Files.lines(Path.of(str), StandardCharsets.UTF\_8) : Stream[String]
	\item Files.write(Path.of(str), lines, StandardCharsets.UTF\_8)
\end{itemize}

\subsection{Stream}
Jeder Stream muss wieder geschlossen werden, ansonsten muss mit Datenverlust gerechnet werden.
\begin{lstlisting}
try (var in = new FileInputStream(path)) {
	int c = in.read();
	while (c >= 0) {
		byte b  = (byte)c;
		
		c = in.read();
	}
}
\end{lstlisting}

\subsubsection{Standard Input/Output}
Folgende Standard Streams werden verwendet um standardmässig auf die Konsole zu Schreiben.
\begin{itemize}[nosep]
	\item System.in
	\item System.out
	\item System.err
\end{itemize}

\subsection{Reader/Writer}
In Java ist der char-Type ein 16Bit (UTF-16) kodierung. Hier einige Beispiele:
\begin{lstlisting}
	try (var in = new FileReader(path)) {
		int c = in.read();
		while (c >= 0) {
			char b  = (char)c;
			
			c = in.read();
		}
	}
\end{lstlisting}

\begin{lstlisting}
	try (var writer = new FileWriter(path, append)) {
		writer.write("Hallo!");
		wirter.write('\n');
	}
\end{lstlisting}

\noindent Für \textbf{Buffered}Reader-/ Writer ist es wichtig, \textit{flush()} aufzurufen. Dies erzwingt den Reader/-Writer alle gepufferten Werte sofort zu Schreiben bzw. Lesen. \textit{close()} ruft impliziet \textit{flush()} auf. Dies wird für Performance-Gründen eingesetzt.
